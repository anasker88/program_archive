\documentclass[a4paper,11pt]{ltjsarticle}


% 数式
\usepackage{amsmath,amsfonts}
\usepackage{bm}
% 画像
\usepackage{graphicx}
%枠付き文書
\usepackage{ascmac}
\usepackage{fancybox}


\begin{document}

\title{知能システム論第10回 問1}
\author{ 05-231001 阿部 桃大}
\maketitle

\section*{問題1}
\begin{itembox}[l]{問題1}
  CKY法とビタビアルゴリズムの計算量を与えよ\newline
  • ⼊⼒⽂の⻑さ: $n$\newline
  • ⾮終端記号の数($|N|$): $k$\newline
  • ⽣成規則の数(|R|): $r$
\end{itembox}
\subsection*{CKY法}
一段目の計算量は$O(nr)$である。

l段目$(l=2,3,\cdots,n)$の計算量は、
\begin{equation*}
  O\left(\sum_{i=0}^{n-l}\sum_{m=i+1}^{i+l-1}k^2 r\right)=O\left(\sum_{i=0}^{n-l} (l-2)k^2 r\right)=O((n-l)lk^2 r)
\end{equation*}
である。

よって、CKY法の計算量は、
\begin{equation*}
  O(nr+\sum_{l=2}^n (n-l)lk^2r)=O(nr+n^3k^2r)=O(n^3k^2r)
\end{equation*}
である。

\subsection*{ビタビアルゴリズム}
ビタビアルゴリズムでは、CKY法の計算に加えて、最尤解を求める処理があるが、この処理は元の計算の定数倍であるので、ビタビアルゴリズムの計算量は、
\begin{equation*}
  O(n^3k^2r)
\end{equation*}
である。

\section*{問題2}
\begin{itembox}[l]{問題2}
  PCFG構⽂解析にA*探索あるいはビームサーチ
  を適⽤するアルゴリズムを構築し、その計算量
  を与えよ。
\end{itembox}
ここでは、ビームサーチを用いたPCFG構文解析のアルゴリズムを示す。

ビームサーチでは、各セルにおいて、入れる非終端記号を制限することで、計算量を削減する。ここでは、各セルにおいて、a個の非終端記号を保持することとする。このとき、アルゴリズムは以下のようになる。ただし、tableはpを優先度とした優先度付きキューである。
\begin{verbatim}
for i = 1 to n
  table (i - 1,i) ← {A| A → x_i ∈ R}
  prob (i - 1,i) ← p(A → x_i)
for l=2 to n
  for i=0 to n-l
    j=i+l
    for m=i+1 to j-1
      for X in table(i,m)
        for Y in table(m,j)
          for Z in N
            if (Z → XY) ∈ R then
              p=prob(i,m,X)*prob(m,j,Y)*p(Z → XY)
              if p>prob(i,j,Z) then
                if |table (i,j)|<a then
                  table (i,j).push(Z)
                  back (i,j,Z) ← (m,X,Y)
                  prob (i,j,Z) ← p
                else
                  if p>min(table (i,j)) then
                    table (i,j).pop()
                    table (i,j).push(Z)
                    back (i,j,Z) ← (m,X,Y)
                    prob (i,j,Z) ← p
\end{verbatim}

このアルゴリズムの計算量は、各セルで探索対象の非終端記号をa個に制限しているので、ビタビアルゴリズムの計算量のkをaに置き換えたものとなる。よって、ビームサーチの計算量は、
\begin{equation*}
  O(n^3a^2r)
\end{equation*}
である。

\end{document}
